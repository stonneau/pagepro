\documentclass{tccv}
\usepackage[french]{babel}
%~ \usepackage[latin1]{inputenc}

\begin{document}

\part{Steve Tonneau}

\section{recherche}

\begin{eventlist}

\item{ Mars 2015 - (Septembre 2016)}
     {LAAS-CNRS, Toulouse, France}
     {Post-Doctorat sur l'ANR Entracte}
     
     L'objectif de ces 18 mois est d'adapter les m\'ethodes de planification de mouvements riches en contact d\'evelop\-p\'ees durant ma th\`ese 
     \`a des robots anthropomorphes tels que HRP-2. Cet objectif va de pair avec une mont\'ee en comp\'etences sur les m\'ethodes d'optimisation et de contr\^ole optimal.
     Du c\^ot\'e de l'animation graphique, coordination d'une collaboration entre le LAAS et l'universit\'e d' Edinbourg sur la synth\`ese en ligne de mouvements de contacts
     dans des environnements dynamiques.

\item{ D\'ecembre 2011 - F\'evrier 2015}
     {IRISA, Rennes, France}
     {Doctorat}
	 
 \textbf{Planification de mouvements pour des personnages virtuels en environnements contraints}.\\
 L'objectif de cette th\`{e}se est d'am\'{e}liorer l'autonomie de mouvement de personnages 3d pour des applications 
 de type jeux vid\'{e}os. Pour ce faire, on cherche \`{a} g\'{e}n\'{e}rer automatiquement des animations qui leur permettent d'interagir de mani\`{e}re
 cr\'{e}\-dible avec des environnements complexes (sortie de v\'ehicule, escalade...).

%~ \item{ D\'ecembre 2011 - F\'evrier 2015}
     %~ {INSA, Rennes, France}
     %~ {Enseignements}

%~ -- Encadrement de projets \'etudiants en r\'ealit\'e virtuelle;
%~ \\-- Programmation fonctionnelle (Scheme) - Cours et TPs;
%~ \\ -- Bases de donn\'{e}es - Cours et TPs;
%~ \\ -- Objective Caml - TPs.

\end{eventlist}


\section{Exp\'{e}rience en industrie (3 ans)}

\begin{eventlist}

\item{Janvier 2010 -- Octobre 2011}
     {Masa Group, Paris, France}
     {Chef de projet: Virtual Worlds}

Projet Form : Animation automatique d'agents virtuels gr\^{a}ce au middleware \href{http://www.masagroup.net/products/masa-life/}{MasaLife}.
Int\'{e}gration avec le syt\`{e}me d'animation \href{http://www.naturalmotion.com/products/morpheme/}{Morpheme} ainsi que l'outil de pathfinding
\href{http://www.presagis.com/products_services/products/modeling-simulation/simulation/aiimplant/}{AI.implant} pour des d\'{e}monstrations techniques.

\item{Juillet 2008 -- D\'{e}cembre 2009}
     {Masa Group, Paris, France}
     {Ing\'{e}nieur R\&T - Mod\'{e}lisateur}

Projet Brain : MiddleWare d'intelligence artificielle pour les serious games (simulations d'entra\^{i}nement).
\\ -- Architecte de la base de connaissances des agents et d'un outil d'analyse du terrain;
\\ -- Op\'{e}rations de maintenance sur le moteur d\'{e}cisionnel;
\\ -- Cr\'{e}ations de librairies de comportements pour diff\'{e}ren\-tes d\'{e}monstrations.


\end{eventlist}


\personal
    [stevetonneau.fr]
    {3 rue Paul Dupin, 31500 Toulouse, France}
    {+33 (0) 671303668}
    {pro@stevetonneau.fr}

\section{Dipl\^omes -- Formation}

\begin{yearlist}
 
\item[Mention Tr\`es honorable]{27/02/15}
     {Doctorat en informatique}
	 {Irisa, Rennes, France}
 
\item[]{Juin 2008 }
     {Dipl\^ome ing\'{e}nieur informatique}
	 {INSA, Rennes, France}
	 
\item[Note A (major)]{2007 -- 2008 }
     {Cours de Master \`a l' \'etranger}
     {RIT, Rochester, USA -- \href{http://games.rit.edu/}{games.rit.edu}}

\item[Mention assez bien]{Juin 2005 }
     {Deug MIAS}
     {Universit\'{e} Montpellier II, France}
	 
\end{yearlist}
	
\section{Comp\'{e}tences informatiques}

\begin{factlist}

\item{D\'eveloppement}
     {HPP, Masalife}

\item{Moteurs}
     {Unity 3d, ODE, Bullet}

\item{3D}
     {Blender}
	 
\item{Languages}
     {C++, Java, C\#, Clojure, Python, Lua}
	 
\item{Robots}
     {HRP-2, HyQ}
	 
\item{Donn\'{e}es}
     {PostgreSQL, PostGIS}

%~ \item{Version}
     %~ {GIT, SVN}

\end{factlist}


% \section{Public projects}

% \begin{yearlist}

% \item{2012}
     % {ntdisp (\href{http://ntdisp.entidi.com/}{ntdisp.entidi.com})}
     % {Embedded devices programmer}

% \item{2007}
     % {tip (\href{http://tip.entidi.com/}{tip.entidi.com})}
     % {PHP framework based on PEAR}

% \item{2006}
     % {adg (\href{http://adg.entidi.com/}{adg.entidi.com})}
     % {Automatic drawing generation}

% \item{2006}
     % {gtk2panel (\href{http://gtk2panel.entidi.com/}{gtk2panel.entidi.com})}
     % {Top panel menu in GTK+2}

% \item{2004}
     % {ntd (\href{http://ntd.entidi.com/}{ntd.entidi.com})}
     % {General purpose libraries}

% \end{yearlist}

\section{Langues}

\begin{factlist}
\item{fluent}{Anglais}
\item{bon}{Portugais}{bon}
\item{Espagnol}{scolaire}%\\ \\ \\ \\  \\ \\ \\ \\  \\ \\ \\ \\ 
\end{factlist}


\section{Responsabilit\'es scientifiques et administratives}
\begin{factlist}
\item{Relecteur}{Conf\'erences SIGGRAPH et SCA. Revue Transaction on Robotics.}
\item{Conf\'erences}{Co-organisateur  MIG 2012 \\ Volontaire Eurographics '13}
\item{Ateliers}{Projet pour ICRA 2016} \\ \\
\end{factlist}

\section{Conf\'erences internationales}
\begin{eventlist}

\item
     {\small{Character contact repositioning under large environment deformation}}{\small{\underline{Tonneau}, Al-Ashqar, Pettr\'e, Komura, Mansard}}{
{\small{Accept\'e (non publi\'e) \`a Eurographics '16}}} 


\item
     {\small{Dynamically Balanced and Plausible Trajectory Planning for Human-Like Characters}}{\small{Park, \underline{Tonneau}, Mansard, Multon, Pettr\'e, Manocha}}{
{\small{Accept\'e (non publi\'e) \`a I3D '16}}} 


\item
     {\small{A Versatile and Efficient Pattern Generator for Generalized Legged Locomotion}}{\small{Carpentier, \underline{Tonneau}, Naveau, Stasse, Mansard}}{
\small{Soumis \`a ICRA '16}} 

\item
     {\small{Fast Algorithms to Test Robust Static Equilibrium for Legged Robots}}{\small{Del Prete, \underline{Tonneau}, Mansard}}{
\small{Soumis \`a ICRA '16}} 

\item
     {\small{A Reachability-based planner for sequences of acyclic contacts in cluttered environments}}{\small{\underline{Tonneau}, Mansard, Park, Multon, Manocha, Pettr\'e}}{
\small{ISRR '15}} 

\item
     {\small{Task efficient contact configurations for arbitrary virtual creatures}}{\small{\underline{Tonneau}, Pettr\'e et Multon}}{
\small{Graphics interface '14}} 
\end{eventlist}
\section{Revues internationales}
\begin{eventlist}

\item
     {\small{An efficient contact planner for multiped robots}}{\small{\underline{Tonneau}, Del Prete, Pettr\'e, Multon et Mansard}}{
\small{Soumis \`a  IJRR}} 

\item
     {\small{Using task efficient contact configurations to animate creatures in arbitrary environments}}{\small{\underline{Tonneau}, Pettr\'e et Multon}}{
\small{Computers \& Graphics vol 45} }

\end{eventlist}

\section{Encadrement (40h enseignant)}
\begin{eventlist}

\item{ Octobre 2015 - }
     {LAAS-CNRS, Toulouse, France}
     {Th\`ese de Pierre Fernbach (avec Michel Ta\"ix)}
     Sujet: G\'en\'eration de mouvements dynamiques riches en contacts
     pour la robotique et l'animation graphique.
     
\item{Janvier 2010 -- Octobre 2011}
     {Masa Group, Paris, France}
     {Chef de projet: Virtual Worlds}
D\'ecoupage des t\^aches du projet, animation des r\'eunions, 
\'evaluation de la qualit\'e du code produit.
     
\item{Octobre 2014 - Juin 2014}
     {INSA, Rennes, France}
     {2 projets \'etudiants en r\'ealit\'e virtuelle (40 h)}
     
     Cr\'eation de jeux \'educatifs sur la th\'eorie de la relativit\'e.


\end{eventlist}

\section{Enseignement (181h)}
\begin{eventlist}
\item{Octobre 2016}
     {Formation entreprises CNRS - 10h}
     {LAAS - Toulouse}
     
Cours de 10h sur les m\'ethodes d'animation pour le jeu vid\'eo, que
j'ai enti\`erement cr\'e\'e (\url{http://cpc.cx/e3x}).

\item{Janvier 2016}
     {Planification de mouvement - 3h}
     {Master 2 - Supaero Toulouse}
     
CM sur mes travaux et la planification de mouvement.

\item{D\'ecembre 2015}
     {Planification de mouvement - 16h}
     {Master 2 - AIP Toulouse}
     
Conception et dispense des TPs.

\item{D\'ecembre 2011 - F\'evrier 2013 (2 sessions)}
     {Programmation fonctionnelle - 48h}
     {Licence 2 - INSA Rennes}
     
Dispense des CM, TD et des TPs. %, pour des etudiants internationaux, non informaticiens.
     

\item{D\'ecembre 2011 - F\'evrier 2014 (3 sessions)}
     {Base de donn\'ees - 84h}
     {Licence 2 - INSA Rennes}
     
Dispense des TD et TPs + encadrement du projet final.

\item{Mars 2012 -- Juin 2012}
     {Ocaml - 20h}
     {Master 1 - INSA Rennes}

	Dispense de TPs. Analyse syntaxique, compilation.

\end{eventlist}


\section{Ref\'erents}
\begin{factlist}
\item{Julien Pettr\'e}{Chercheur, Inria Rennes, France. jpettre@inria.fr}
\item{Franck Multon}{Directeur de recherche, Irisa Rennes, France. fmulton@irisa.fr}
\item{Nicolas Mansard}{Chercheur, LAAS-CNRS, Toulouse, France. nmansard@laas.fr}
\item{Jean-Paul Laumond}{Directeur de recherche, LAAS-CNRS, Toulouse, France. jpl@laas.fr}
\item{Dinesh Manocha}{Professeur, chef d'\'equipe Gamma, UNC, Chapel hill, USA. dm@cs.unc.edu}
\item{Taku Komura}{Professeur, University of Edinburg, Ecosse. tkomura@ed.ac.uk}
\end{factlist}

%~ \item{ D\'ecembre 2011 - F\'evrier 2015}
     %~ {INSA, Rennes, France}
     %~ {Enseignements}

%~ -- Encadrement de projets \'etudiants en r\'ealit\'e virtuelle;
%~ \\-- Programmation fonctionnelle (Scheme) - Cours et TPs;
%~ \\ -- Bases de donn\'{e}es - Cours et TPs;
%~ \\ -- Objective Caml - TPs.

%~ \section{Autres}
%~ \begin{factlist}
%~ 
%~ \item{Sport}
     %~ {Rugby, Basket-Ball, Course \`{a} pied}
	 %~ 
%~ \item{Voyages}
     %~ {Br\'{e}sil (3 ans), Etats-unis (6 mois), Canada, Mali, Cap Vert, Europe...}
	 %~ 
%~ \item{Loisirs}
     %~ {Cin\'ema, bande dessin\'{e}e, piano, danse (\href{http://www.youtube.com/watch?v=7R2Pxj9kCdk}{lindy hop)}}
%~ 
%~ \end{factlist}


\end{document}
