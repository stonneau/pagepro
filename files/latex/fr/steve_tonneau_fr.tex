\documentclass{tccv}
\usepackage[french]{babel}

\begin{document}

\part{Steve Tonneau}

\section{recherche - enseignement}

\begin{eventlist}

\item{ Mars 2015 - (Septembre 2016)}
     {LAAS-CNRS, Toulouse, France}
     {Post-Doctorat sur l'ANR Entracte}
     
     L'objectif de ces 18 mois est d'adapter les m\'ethodes de planification de mouvements riches en contact d\'evelop\-p\'ees durant ma th\`ese 
     \`a des robots anthropomorphes tels que HRP-2. Cet objectif va de pair avec une mont\'ee en comp\'etences sur les m\'ethodes d'optimisation et de contr\^ole optimal.
     Du c\^ot\'e de l'animation graphique, coordination d'une collaboration entre le LAAS et l'universit\'e d' Edinbourg sur la synth\`ese en ligne de mouvements de contacts
     dans des environnements dynamiques.

\item{ D\'ecembre 2011 - F\'evrier 2015}
     {IRISA, Rennes, France}
     {Doctorat}
	 
 \textbf{Planification de mouvements pour des personnages virtuels en environnements contraints}.\\
 L'objectif de cette th\`{e}se est d'am\'{e}liorer l'autonomie de mouvement de personnages 3d pour des applications 
 de type jeux vid\'{e}os. Pour ce faire, on cherche \`{a} g\'{e}n\'{e}rer automatiquement des animations qui leur permettent d'interagir de mani\`{e}re
 cr\'{e}\-dible avec des environnements complexes (sortie de v\'ehicule, escalade...).

\item{ D\'ecembre 2011 - F\'evrier 2015}
     {INSA, Rennes, France}
     {Enseignements}

-- Encadrement de projets \'etudiants en r\'ealit\'e virtuelle;
\\-- Programmation fonctionnelle (Scheme) - Cours et TPs;
\\ -- Bases de donn\'{e}es - Cours et TPs;
\\ -- Objective Caml - TPs.

\end{eventlist}


\section{Exp\'{e}rience en industrie (3 ans)}

\begin{eventlist}

\item{Janvier 2010 -- Octobre 2011}
     {Masa Group, Paris, France}
     {Chef de projet: Virtual Worlds}

Projet Form : Animation automatique d'agents virtuels gr\^{a}ce au middleware \href{http://www.masagroup.net/products/masa-life/}{MasaLife}.
Int\'{e}gration avec le syt\`{e}me d'animation \href{http://www.naturalmotion.com/products/morpheme/}{Morpheme} ainsi que l'outil de pathfinding
\href{http://www.presagis.com/products_services/products/modeling-simulation/simulation/aiimplant/}{AI.implant} pour des d\'{e}monstrations techniques.

\item{Juillet 2008 -- D\'{e}cembre 2009}
     {Masa Group, Paris, France}
     {Ing\'{e}nieur R\&T - Mod\'{e}lisateur}

Projet Brain : MiddleWare d'intelligence artificielle pour les serious games (simulations d'entra\^{i}nement).
\\ -- Architecte de la base de connaissances des agents et d'un outil d'analyse du terrain;
\\ -- Op\'{e}rations de maintenance sur le moteur d\'{e}cisionnel;
\\ -- Cr\'{e}ations de librairies de comportements pour diff\'{e}ren\-tes d\'{e}monstrations.


\end{eventlist}


\personal
    [stevetonneau.fr]
    {3 rue Paul Dupin, 31500 Toulouse, France}
    {+33 (0) 671303668}
    {stevetonneau@hotmail.fr}

\section{Formation}

\begin{yearlist}
 
\item[INSA, Rennes, France]{2005 -- 2008 }
     {Dipl\^ome ing\'{e}nieur informatique}
	 
\item[Semestre \`{a} l'\'{e}tranger]{2005 -- 2008 }
     {Master "Game design and Development"}
     {RIT, Rochester, USA -- \href{http://games.rit.edu/}{games.rit.edu}}

\item[Deug mias]{2003 -- 2005 }
     {Math\'{e}matiques et informatique}
     {Universit\'{e} Montpellier II, France}
	 
\end{yearlist}
	
\section{Comp\'{e}tences informatiques}

\begin{factlist}

\item{Moteurs}
     {Unity 3d, Morpheme, ODE, Bullet}
	 
\item{Languages}
     {C++, Java, C\#, Prolog, Lua, Scheme, O-Caml}
	 
\item{Donn\'{e}es}
     {PostgreSQL, PostGIS}

\item{Version}
     {GIT, SVN}

\end{factlist}


% \section{Public projects}

% \begin{yearlist}

% \item{2012}
     % {ntdisp (\href{http://ntdisp.entidi.com/}{ntdisp.entidi.com})}
     % {Embedded devices programmer}

% \item{2007}
     % {tip (\href{http://tip.entidi.com/}{tip.entidi.com})}
     % {PHP framework based on PEAR}

% \item{2006}
     % {adg (\href{http://adg.entidi.com/}{adg.entidi.com})}
     % {Automatic drawing generation}

% \item{2006}
     % {gtk2panel (\href{http://gtk2panel.entidi.com/}{gtk2panel.entidi.com})}
     % {Top panel menu in GTK+2}

% \item{2004}
     % {ntd (\href{http://ntd.entidi.com/}{ntd.entidi.com})}
     % {General purpose libraries}

% \end{yearlist}

\section{Langues}

\begin{factlist}
\item{Anglais}{fluent}
\item{Portugais}{bon}
\item{Espagnol}{scolaire}
\end{factlist}

\section{Publications}
\begin{factlist}

\item{Conf.} 
     {\textbf{Task efficient contact configurations for 
arbitrary virtual creatures.} \\
\underline{Tonneau}, Pettr\'e et Multon \\
\textit{Graphics interface ’14, conference }
}

\item{Conf.} 
     {\textbf{A Reachability-based planner for sequences
of acyclic contacts in cluttered environments.} \\
\underline{Tonneau}, Mansard, Park, Manocha, Multon et Pettr\'e\\
\textbf{submitted to} \textit{ISRR '15}
}


\item{Journal} 
     {\textbf{Using task efficient contact configurations to animate 
creatures in arbitrary environments.} \\
\underline{Tonneau}, Pettr\'e et Multon \\
\textit{Computers \& Graphics vol 45}
}

\end{factlist}

%~ \section{Autres}
%~ \begin{factlist}
%~ 
%~ \item{Sport}
     %~ {Rugby, Basket-Ball, Course \`{a} pied}
	 %~ 
%~ \item{Voyages}
     %~ {Br\'{e}sil (3 ans), Etats-unis (6 mois), Canada, Mali, Cap Vert, Europe...}
	 %~ 
%~ \item{Loisirs}
     %~ {Cin\'ema, bande dessin\'{e}e, piano, danse (\href{http://www.youtube.com/watch?v=7R2Pxj9kCdk}{lindy hop)}}
%~ 
%~ \end{factlist}


\end{document}
