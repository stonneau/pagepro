\documentclass{tccv}
\usepackage[english]{babel}

\begin{document}

\part{Steve Tonneau}

\section{Research experience}

\begin{eventlist}

\item{ March 2015 - (September 2016)}
     {LAAS-CNRS, Toulouse, France}
     {Post-Doctorate -- national project Entracte (ANR)}
     
     The objective of those 18 months is to adapt the multi contact planning methods I developed 
     during my PhD from Computer animation to anthropomorphic robots such as HRP-2.
     This implies developping skills regarding optimization and optimal control methods, in collaboration with \href{http://projects.laas.fr/gepetto/index.php/Members/NicolasMansard}{Nicolas Mansard}.
     In parallel, I pursue my Computer Graphics activity by coordinating a new collaboration between LAAS and The University of Edinburgh on online
     synthesis of contact rich motions in dynamic environments.

\item{December 2011 - February 2015}
     {IRISA, Rennes, France}
     {Phd}
	 
 \textbf{Autonomous locomotion for virtual characters in constrained environments}.
 This thesis objective is to improve the motion autonomy of 3d characters in applications such as videogames.
We therefore propose new methods and heuristics to generate the animations that allow them to interact with complex environments in a believable manner (car outgress, climbing tasks...).


\item{December 2011 - February 2015}
     {INSA, Rennes, France}
     {Teachings}

-- Supervising of student projects on Virtual Reality;
\\-- Functional programming (Scheme);
\\ -- Databases;
\\ -- Objective Caml programming language.

\end{eventlist}


\section{Engineer experience (3.5 Years)}

\begin{eventlist}

\item{January 2010 -- October 2011}
     {Masa Group, Paris, France}
     {Project Manager - Virtual Worlds}

Form project : AI-driven animation for 3d characters using the \href{http://www.masagroup.net/products/masa-life/}{MasaLife} AI middleware.
Integration with \href{http://www.naturalmotion.com/products/morpheme/}{Morpheme} animation framework and
\href{http://www.presagis.com/products_services/products/modeling-simulation/simulation/aiimplant/}{AI.implant} pathfinding
solution within technical demonstrations.

\item{July 2008 -- December 2009}
     {Masa Group, Paris, France}
     {R\&T Engineer - Behavior modeller}

Brain project : Artificial Intelligence middleware for serious games.
\\ -- Knowledge base designer;
\\ -- Maintenance on the decisional engine;
\\ -- Behavior libraries design for various demonstrations.

\end{eventlist}


\personal
    [stevetonneau.fr]
    {3 rue Paul Dupin, 31500 Toulouse, France}
    {+33 (0) 671303668}
    {stevetonneau@hotmail.fr}

\section{Education}

\begin{yearlist}

 \item[INSA Engineering school ]{2005 -- 2008}
     {Master in Computer science}
     {Rennes, France}
 
\item[Abroad semester at RIT]{2005 -- 2008 }
     {"Game design and Development" Master classes}
     {Rochester, USA}

\item[University of Montpellier II]{2003 -- 2005}
	 {2 year diploma on Mathematics}
     {Montpellier, France}
	 
% \item[High school diploma]{1988 -- 1992}
     % {Informatic engineer}
     % {ITIS Castelli, Brescia}

% \item{1987 -- 1988}
     % {Classical gymnasium}
     % {Seminario vescovile, Cremona}

\end{yearlist}

\section{Computer science skills}

\begin{factlist}

\item{Engines}
     {Unity 3d, Morpheme, ODE, Bullet}
	 
\item{Languages}
     {C++, Java, C\#, Prolog, Lua, Scheme, O-Caml}
	 
\item{Data}
     {PostgreSQL, PostGIS}

\item{Version control}
     {GIT, SVN}

\end{factlist}



% \section{Public projects}

% \begin{yearlist}

% \item{2012}
     % {ntdisp (\href{http://ntdisp.entidi.com/}{ntdisp.entidi.com})}
     % {Embedded devices programmer}

% \item{2007}
     % {tip (\href{http://tip.entidi.com/}{tip.entidi.com})}
     % {PHP framework based on PEAR}

% \item{2006}
     % {adg (\href{http://adg.entidi.com/}{adg.entidi.com})}
     % {Automatic drawing generation}

% \item{2006}
     % {gtk2panel (\href{http://gtk2panel.entidi.com/}{gtk2panel.entidi.com})}
     % {Top panel menu in GTK+2}

% \item{2004}
     % {ntd (\href{http://ntd.entidi.com/}{ntd.entidi.com})}
     % {General purpose libraries}

% \end{yearlist}

\section{Communication skills}

\begin{factlist}
\item{English}{fluent}
\item{Portuguese}{good}
\item{Spanish}{fair}
\end{factlist}

\section{Publications}
\begin{factlist}

\item{Conf.} 
     {\textbf{Task efficient contact configurations for 
arbitrary virtual creatures.} \\
\underline{Tonneau}, Pettr\'e et Multon \\
\textit{Graphics interface ’14, conference }
}

\item{Conf.} 
     {\textbf{A Reachability-based planner for sequences
of acyclic contacts in cluttered environments.} \\
\underline{Tonneau}, Mansard, Park, Manocha, Multon et Pettr\'e\\
\textbf{submitted to} \textit{ISRR '15}
}


\item{Journal} 
     {\textbf{Using task efficient contact configurations to animate 
creatures in arbitrary environments.} \\
\underline{Tonneau}, Pettr\'e et Multon \\
\textit{Computers \& Graphics vol 45}
}

\end{factlist}


\end{document}
