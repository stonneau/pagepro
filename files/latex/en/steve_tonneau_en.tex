\documentclass{tccv}
\usepackage[english]{babel}

\begin{document}

\part{Steve Tonneau}

\section{Keywords}
\textbf{Motion planning and synthesis, optimal control, humanoid robotics, procedural animation}

\section{Fundings and awards}
\textbf{Digital Technology Grand Prize (ANR Entracte). \\ Royal Society International Exchanges Award.}

\section{Research experience}
\begin{eventlist}
\item{(Upcoming) December 2017 - March 2018}
     {The University of Edinburgh, Scotland}
     {Post-Doctorate in mobility -- Royal Society International Exchanges Award}
     
     Long stay in Edinburg. On this new project, we will work on deep learning methods and their applications for motion synthesis
     in both robotics and computer graphics. 

\item{Since March 2015 }
     {LAAS-CNRS, Toulouse, France}
     {Post-Doctorate -- national project Entracte (ANR)}
     
     The objective of the post-doc is to adapt the multi contact planning methods I developed 
     during my PhD from Computer animation to anthropomorphic robots such as HRP-2.
     This implies developping skills regarding optimization and optimal control methods, in collaboration with \href{http://projects.laas.fr/gepetto/index.php/Members/NicolasMansard}{Nicolas Mansard} and 
     \href{https://andreadelprete.github.io/}{Andrea Del Prete}.
     In parallel, I pursue my Computer Graphics activity by coordinating a new collaboration between LAAS and The University of Edinburgh on online
     synthesis of contact rich motions in dynamic environments with Prof. \href{http://homepages.inf.ed.ac.uk/tkomura/}{Taku Komura}.

\item{December 2011 - February 2015}
     {IRISA, Rennes, France}
     {Phd}
	 
 \textbf{Autonomous locomotion for virtual characters in constrained environments}.
 This thesis objective is to improve the motion autonomy of 3d characters in applications such as videogames.
We therefore propose new methods and heuristics to generate the animations that allow them to interact with complex environments in a believable manner (car outgress, climbing tasks...).



\item{Since December 2011}
     {INSA Rennes, Supaero Toulouse,\\ AIP Toulouse, France}
     {Teachings}
     
%~ -- Motion planning
%~ \\-- Linear algebra$^1$
-- humanoid robotics\footnote{with my own class material}: motion planning and control
\\-- Supervising of student projects on Virtual Reality;
\\-- Functional programming (Scheme);
\\ -- Databases;
\\ -- Objective Caml programming language.

%~ \item{December 2011 - February 2015}
     %~ {INSA, Rennes, France}
     %~ {Teachings}
%~ 
%~ -- Supervising of student projects on Virtual Reality;
%~ \\-- Functional programming (Scheme);
%~ \\ -- Databases;
%~ \\ -- Objective Caml programming language.

\end{eventlist}

\personal
    [stevetonneau.fr]
    {250 avenue de Muret, 31300 Toulouse, France}
    {+33 (0) 671303668}
    {stevetonneau@hotmail.fr}


\section{Supervision}

\begin{factlist}
\item{Pierre Fernbach}{PhD student}
\item{Rapha\"el Lef\`evre}{Intern}
\item{Anna Sepp\"ala}{Software engineer}
\end{factlist}


\section{Engineer experience (3.5 Years)}

\begin{eventlist}

\item{July 2008 -- October 2011}
     {Masa Group, Paris, France}
     {Software Engineer then Project Manager}

%~ As a research engineer, I have been in charge of two projects.
\textbf{Brain project} : AI middleware for serious games.
\\ -- Knowledge base designer;
\\ -- Maintenance on the decisional engine;
\\ -- Behavior libraries design for various demonstrations.
\textbf{Form project} : AI-driven animation for 3d characters using the \href{http://www.craft.ai/}{MasaLife} AI middleware (now craftai).
Integration with \href{http://www.naturalmotion.com/products/morpheme/}{Morpheme} animation framework and
AI.implant pathfinding
solution within technical demonstrations.

%~ \item{July 2008 -- December 2009}
     %~ {Masa Group, Paris, France}
     %~ {R\&T Engineer - Behavior modeller}


\end{eventlist}



\section{Education}

\begin{yearlist}

 \item[INSA Engineering school ]{2005 -- 2008}
     {Master in Computer science}
     {Rennes, France}
 
\item[Abroad semester at RIT]{2005 -- 2008}
     {``Game design and Development'' Master classes}
     {Rochester, USA}

\item[University of Montpellier II]{2003 -- 2005}
	 {2 year diploma on Mathematics}
     {Montpellier, France}
	 
% \item[High school diploma]{1988 -- 1992}
     % {Informatic engineer}
     % {ITIS Castelli, Brescia}

% \item{1987 -- 1988}
     % {Classical gymnasium}
     % {Seminario vescovile, Cremona}

\end{yearlist}

\section{Computer science skills}

\begin{factlist}

\item{Engines and software}
     {Blender, Unity 3d, Morpheme, ODE, Bullet}
	 
\item{Languages}
     {C++, Python, Java, C\#, Prolog, Lua, Scheme, O-Caml}
	 
\item{Data}
     {PostgreSQL, PostGIS}

\item{Version control}
     {GIT, SVN}

\end{factlist}



% \section{Public projects}

% \begin{yearlist}

% \item{2012}
     % {ntdisp (\href{http://ntdisp.entidi.com/}{ntdisp.entidi.com})}
     % {Embedded devices programmer}

% \item{2007}
     % {tip (\href{http://tip.entidi.com/}{tip.entidi.com})}
     % {PHP framework based on PEAR}

% \item{2006}
     % {adg (\href{http://adg.entidi.com/}{adg.entidi.com})}
     % {Automatic drawing generation}

% \item{2006}
     % {gtk2panel (\href{http://gtk2panel.entidi.com/}{gtk2panel.entidi.com})}
     % {Top panel menu in GTK+2}

% \item{2004}
     % {ntd (\href{http://ntd.entidi.com/}{ntd.entidi.com})}
     % {General purpose libraries}

% \end{yearlist}

%~ \section{Communication skills}
%~ 
%~ \begin{factlist}
%~ \item{English}{fluent}
%~ \item{Portuguese}{good}
%~ \item{Spanish}{fair}
%~ \end{factlist}

\section{Workshops}
%~ \begin{factlist}
%~ \item{Workshop}{}
%~ \end{factlist}
\begin{eventlist}
\item{IROS '16}
{Towards a unified workflow for multi contact motion on legged robots:
Challenges in planning, optimization and control}
{\underline{Tonneau}, Bretl, Mansard}

Main organizer of the workshop (\href{http://homepages.laas.fr/nmansard/entracte/index.php?n=Publication.WorkshopIROS2016}{link}).
\end{eventlist}

\section{reviewer}
\textbf{Regular reviewer for CGF, T-RO, MIG, ICRA and IROS.}
%~ \end{eventlist}


\section{Conference Publications}

\begin{eventlist}

\item{IROS '17}
{ A Kinodynamic steering-method for legged multi-contact locomotion}
{Fernbach (my PhD student), \underline{Tonneau}, Mansard, Park, Manocha, Multon, Pettr\'e}

\item{MIG '16}
{Ballistic motion planning for jumping superheroes}
{Campana, Fernbach, \underline{Tonneau}, Ta\"ix, Laumond}

\item{IROS '16}
{HPP: a new software for constrained motion planning }
{Mirabel, \underline{Tonneau}, Fernbach, Sepp\"ala, Campana, Mansard, Lamiraux}

\item{ICRA '16}
{Fast Algorithms to Test Robust Static Equilibrium for Legged Robots}
{Del Prete, \underline{Tonneau},  Mansard}

\item{ICRA '16}
{A versatile and efficient pattern generator for generalized legged locomotion}
{Carpentier, \underline{Tonneau}, Naveau, Stasse,  Mansard}

\item{ISRR '15}
{A Reachability-based planner for sequences
of acyclic contacts in cluttered environments}
{\underline{Tonneau}, Mansard, Park, Manocha, Multon, Pettr\'e}
%~ {ISRR '15}

\item
 {Graphics interface '14}
 {Task efficient contact configurations for 
arbitrary virtual creatures}
 {\underline{Tonneau}, Pettr\'e et Multon} 


\end{eventlist}

\section{Journal Publications}

\begin{eventlist}

\item {Submitted to TOG} 
{2PAC: Two Point Attractors for Center of Mass
Trajectories in Multi Contact Scenarios}
{\underline{Tonneau}, Del Prete, Pettr\'e, Mansard}

\item {Submitted to T-RO} 
{An efficient acyclic contact planner for multiped robots}
{\underline{Tonneau}, Del Prete, Pettr\'e, Park, Manocha, Mansard}

\item {Conditionally accepted to T-RO} 
{Zero Step Capturability for Legged Robots in Multi Contact}
{Del Prete, \underline{Tonneau}, Mansard}

\item {Computer Graphics Forum (Eurographics '16)} 
{Character contact re-positioning under large environment deformation}
{\underline{Tonneau}, Al-Ashqar, Pettr\'e, Komura, Mansard}

\item {Computers \& Graphics vol 45} 
{Using task efficient contact configurations to animate 
creatures in arbitrary environments}
{\underline{Tonneau}, Pettr\'e, Multon}



\end{eventlist}

%~ \begin{factlist}
%~ 
%~ \item{Conf.} 
     %~ {\textbf{Task efficient contact configurations for 
%~ arbitrary virtual creatures.} \\
%~ \underline{Tonneau}, Pettr\'e, Multon \\
%~ \textit{Graphics interface ’14, conference }
%~ }
%~ 
%~ \item{Conf.} 
     %~ {\textbf{A Reachability-based planner for sequences
%~ of acyclic contacts in cluttered environments.} \\
%~ \underline{Tonneau}, Mansard, Park, Manocha, Multon, Pettr\'e\\
%~ \textbf{submitted to} \textit{ISRR '15}
%~ }
%~ 
%~ 
%~ \item{Journal} 
     %~ {\textbf{Using task efficient contact configurations to animate 
%~ creatures in arbitrary environments.} \\
%~ \underline{Tonneau}, Pettr\'e, Multon \\
%~ \textit{Computers \& Graphics vol 45}
%~ }
%~ 
%~ \end{factlist}


\end{document}
